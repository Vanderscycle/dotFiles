%%%%%%%%%%%%%%%%%
% This is an sample CV template created using altacv.cls
% (v1.7.2, 28 August 2024) written by LianTze Lim (liantze@gmail.com). Compiles with pdfLaTeX, XeLaTeX and LuaLaTeX.
%
%% It may be distributed and/or modified under the
%% conditions of the LaTeX Project Public License, either version 1.3
%% of this license or (at your option) any later version.
%% The latest version of this license is in
%%    http://www.latex-project.org/lppl.txt
%% and version 1.3 or later is part of all distributions of LaTeX
%% version 2003/12/01 or later.
%%%%%%%%%%%%%%%%

%% Use the "normalphoto" option if you want a normal photo instead of cropped to a circle
% \documentclass[10pt,a4paper,normalphoto]{altacv}

\documentclass[10pt,a4paper,ragged2e,withhyper]{altacv}
%% AltaCV uses the fontawesome5 and simpleicons packages.
%% See http://texdoc.net/pkg/fontawesome5 and http://texdoc.net/pkg/simpleicons for full list of symbols.

% Change the page layout if you need to
\geometry{left=1.25cm,right=1.25cm,top=1.5cm,bottom=1.5cm,columnsep=1.2cm}

% The paracol package lets you typeset columns of text in parallel
\usepackage{paracol}

% Change the font if you want to, depending on whether
% you're using pdflatex or xelatex/lualatex
% WHEN COMPILING WITH XELATEX PLEASE USE
% xelatex -shell-escape -output-driver="xdvipdfmx -z 0" sample.tex
\ifxetexorluatex
  % If using xelatex or lualatex:
  \setmainfont{Roboto Slab}
  \setsansfont{Lato}
  \renewcommand{\familydefault}{\sfdefault}
\else
  % If using pdflatex:
  \usepackage[rm]{roboto}
  \usepackage[defaultsans]{lato}
  % \usepackage{sourcesanspro}
  \renewcommand{\familydefault}{\sfdefault}
\fi

% catppuccin mocha
\definecolor{Rosewater}{HTML}{f5e0dc}
\definecolor{Flamingo}{HTML}{f2cdcd}
\definecolor{Pink}{HTML}{f5c2e7}
\definecolor{Mauve}{HTML}{cba6f7}
\definecolor{Red}{HTML}{f38ba8}
\definecolor{Maroon}{HTML}{eba0ac}
\definecolor{Peach}{HTML}{fab387}
\definecolor{Yellow}{HTML}{f9e2af}
\definecolor{Green}{HTML}{a6e3a1}
\definecolor{Teal}{HTML}{94e2d5}
\definecolor{Sky}{HTML}{89dceb}
\definecolor{Sapphire}{HTML}{74c7ec}
\definecolor{Blue}{HTML}{89b4fa}
\definecolor{Lavender}{HTML}{b4befe}
\definecolor{Text}{HTML}{cdd6f4}
\definecolor{Subtext1}{HTML}{bac2de}
\definecolor{Subtext0}{HTML}{a6adc8}
\definecolor{Overlay2}{HTML}{9399b2}
\definecolor{Overlay1}{HTML}{7f849c}
\definecolor{Overlay0}{HTML}{6c7086}
\definecolor{Surface2}{HTML}{585b70}
\definecolor{Surface1}{HTML}{45475a}
\definecolor{Surface0}{HTML}{313244}
\definecolor{Base}{HTML}{1e1e2e}
\definecolor{Mantle}{HTML}{181825}
\definecolor{Crust}{HTML}{11111b}

\colorlet{name}{Text}          % For the primary text color (light tone)
\colorlet{tagline}{Red}        % A vibrant red tone for taglines
\colorlet{heading}{Maroon}     % A darker red, for headings
\colorlet{headingrule}{Peach}  % A warm accent for the heading rule, replacing GoldenEarth
\colorlet{subheading}{Red}     % A softer red for subheadings
\colorlet{accent}{Sapphire}    % A cool accent, bright but not overpowering
\colorlet{emphasis}{Overlay1}  % A subdued grey for emphasis, not too bold
\colorlet{body}{Subtext1}      % A soft, neutral tone for the body text
\pagecolor{Base}
% Change some fonts, if necessary
\renewcommand{\namefont}{\Huge\rmfamily\bfseries}
\renewcommand{\personalinfofont}{\footnotesize}
\renewcommand{\cvsectionfont}{\LARGE\rmfamily\bfseries}
\renewcommand{\cvsubsectionfont}{\large\bfseries}


% Change the bullets for itemize and rating marker
% for \cvskill if you want to
\renewcommand{\cvItemMarker}{{\small\textbullet}}
\renewcommand{\cvRatingMarker}{\faCircle}
% ...and the markers for the date/location for \cvevent
% \renewcommand{\cvDateMarker}{\faCalendar*[regular]}
% \renewcommand{\cvLocationMarker}{\faMapMarker*}


% If your CV/résumé is in a language other than English,
% then you probably want to change these so that when you
% copy-paste from the PDF or run pdftotext, the location
% and date marker icons for \cvevent will paste as correct
% translations. For example Spanish:
% \renewcommand{\locationname}{Ubicación}
% \renewcommand{\datename}{Fecha}


%% Use (and optionally edit if necessary) this .tex if you
%% want to use an author-year reference style like APA(6)
%% for your publication list
% \input{pubs-authoryear.cfg}

%% Use (and optionally edit if necessary) this .tex if you
%% want an originally numerical reference style like IEEE
%% for your publication list
\usepackage[backend=biber,style=ieee,sorting=ydnt,defernumbers=true]{biblatex}
%% For removing numbering entirely when using a numeric style
\setlength{\bibhang}{1.25em}
\DeclareFieldFormat{labelnumberwidth}{\makebox[\bibhang][l]{\itemmarker}}
\setlength{\biblabelsep}{0pt}
\defbibheading{pubtype}{\cvsubsection{#1}}
\renewcommand{\bibsetup}{\vspace*{-\baselineskip}}
\AtEveryBibitem{%
  \iffieldundef{doi}{}{\clearfield{url}}%
}

%% sample.bib contains your publications
\addbibresource{sample.bib}

\begin{document}
\name{Henri Vandersleyen}
\tagline{Software Engineer}
%% You can add multiple photos on the left or right
%%\photoR{2.8cm}{Globe_High}
% \photoL{2.5cm}{Yacht_High,Suitcase_High}

\personalinfo{%
  % Not all of these are required!
  \email{henri-vandersleyen@protonmail.com}
  % \phone{000-00-0000}
  % \mailaddress{Åddrésş, Street, 00000 Cóuntry}
  \location{Victoria, BC, Canada}
  \homepage{me.vandersleyen.dev}
  % \twitter{@twitterhandle}
  % \xtwitter{@x-handle}
  \linkedin{/henri-vandersleyen-a25a8312b}
  \github{vanderscycle}
  % \orcid{0000-0000-0000-0000}
  %% You can add your own arbitrary detail with
  %% \printinfo{symbol}{detail}[optional hyperlink prefix]
  % \printinfo{\faPaw}{Hey ho!}[https://example.com/]

  %% Or you can declare your own field with
  %% \NewInfoFiled{fieldname}{symbol}[optional hyperlink prefix] and use it:
  % \NewInfoField{gitlab}{\faGitlab}[https://gitlab.com/]
  % \gitlab{your_id}
  %%
  %% For services and platforms like Mastodon where there isn't a
  %% straightforward relation between the user ID/nickname and the hyperlink,
  %% you can use \printinfo directly e.g.
  % \printinfo{\faMastodon}{@username@instace}[https://instance.url/@username]
  %% But if you absolutely want to create new dedicated info fields for
  %% such platforms, then use \NewInfoField* with a star:
  % \NewInfoField*{mastodon}{\faMastodon}
  %% then you can use \mastodon, with TWO arguments where the 2nd argument is
  %% the full hyperlink.
  % \mastodon{@username@instance}{https://instance.url/@username}
}

\makecvheader
%% Depending on your tastes, you may want to make fonts of itemize environments slightly smaller
% \AtBeginEnvironment{itemize}{\small}

%% Set the left/right column width ratio to 6:4.
\columnratio{0.6}

% Start a 2-column paracol. Both the left and right columns will automatically
% break across pages if things get too long.
\begin{paracol}{2}
\cvsection{Experience}
\cvevent{DevOps Engineer}{Operto Guest Technologies}{Jul 2023 --
  Ongoing}{Vancouver, BC}
\begin{itemize}
Migrated legacy java applications; increased application reliability and
visibility. Used IaC to control our AWS accounts; reduced costs, and increased
performance. Performed Security audit of our accounts; cloud infrastructure
inline with AWS Startup Security Baseline (SSB)
\end{itemize}
\cvtag{AWS}
\cvtag{Terraform}
\cvtag{Java}

\divider

\cvevent{DevOps Engineer}{Atreides}{Apr 2022 -- Jul 2023}{Victoria, BC}
\begin{itemize}
Assisted the Senior DevOps Engineer with managing, adding and deploying
Kubernetes applications to AWS EKS using GitOps pattern. Created and updated
Github CICD pipelines to manage processes.
\end{itemize}
\cvtag{Kubernetse}
\cvtag{Helm}
\cvtag{ArgoCD}
\cvtag{GitOps}
\cvtag{AWS}
\cvtag{Terraform}

\divider

\cvevent{Jr. FullStack Software Developer}{Aviary Tech}{Jul 2021 -- Feb 2022}{Victoria, BC}
\begin{itemize}
As a Jr. Fullstack Software Developer at a startup, I created, tested, and
deployed several blockchain focused fullstack applications following Test-Driven
Development (TDD). Designed smart UI/UX features to enhance user engagement and
increased reliability of backend REST APIs.
\end{itemize}

\cvtag{NextJs}
\cvtag{Svelte}
\cvtag{Jest}

\divider

\cvevent{Marine Engineering Officer}{Royal Canadian Navy}{Apr 2012 -- Sept 2020}{Canada}
\begin{itemize}
I was Responsible for the readiness, operation and maintenance of propulsion
  and ancillary systems, power generation and distribution, hull structure,
  ship’s stability, and damage control. Advised the ship's captain on technical
  readiness and repairs. While in port, routinely held charge of the ship's
  operation and safety as the Officer Of the Day.
\end{itemize}

\cvtag{leadership}
\cvtag{management}
\cvtag{maintenance}
\cvtag{fitness}


%\cvsection{A Day of My Life}

% Adapted from @Jake's answer from http://tex.stackexchange.com/a/82729/226
% \wheelchart{outer radius}{inner radius}{
% comma-separated list of value/text width/color/detail}
% \wheelchart{1.5cm}{0.5cm}{%
%   6/8em/accent!30/{Sleep,\\beautiful sleep},
%   3/8em/accent!40/Hopeful novelist by night,
%   8/8em/accent!60/Daytime job,
%   2/10em/accent/Sports and relaxation,
%   5/6em/accent!20/Spending time with family
% }

% use ONLY \newpage if you want to force a page break for
% ONLY the current column
%\newpage

\nocite{*}

%% Switch to the right column. This will now automatically move to the second
%% page if the content is too long.
\switchcolumn

\cvsection{Skills}
\cvtag{python}
\cvtag{go}
\cvtag{SvelteKit}
\cvtag{TypeScript}

\divider

\cvtag{Terraform}
\cvtag{Nix}
\cvtag{bash}

\divider

\cvtag{Git}
\cvtag{Emacs}
\cvtag{Linux}

\divider

\cvtag{French}
\cvtag{English}
\cvtag{Chinese}
%% Yeah I didn't spend too much time making all the
%% spacing consistent... sorry. Use \smallskip, \medskip,
%% \bigskip, \vspace etc to make adjustments.
\medskip

\cvsection{Education}

\cvevent{Diploma Data Science}{Lighthouse Labs}{Sept 2002 -- Dec 2020}{}

\divider

\cvevent{B.eng.\ Electrical Engineering}{Royal Military College of Canada}{Sept 2007 -- Apr 2012}{}

\cvsection{Achievements}

\cvachievements{\faCertificate}
Earned Head of department certification

\divider

\cvachievements{\faCertificate}
Earned P.eng Certification: 221232

\divider

\cvachievements{\faCertificate}
Qualified as Naval diving supervisor and Firefighting team attack leader

\cvsection{Projects}

\cvevent{Infrastructure}{}{ongoing}{}
\begin{itemize}
\item Linode cloud, kubernetes hosted infrastructure using Helm, Kustomize and
  ArgoCD.
\item \faGithub Vanderscycle/infrastructure
\end{itemize}

\cvevent{Professional website}{}{ongoing}{}
\begin{itemize}
\item Simple SvelteKit website hosted on my kubernetes cluster.
\item \faGithub Vanderscycle/Professional-website
\end{itemize}
\end{paracol}


\end{document}
